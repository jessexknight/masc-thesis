%%%%%%%%%%%%%%%%%%%%%%%%%%%%%%%%%%%%%%%%%%%%%%%%%%%%%%%%%%%%%%%%%%%%%%%%%%%%%%%%%%%%%%%%%%%%%%%%%%%%
% ==================================================================================================
% --------------------------------------------------------------------------------------------------
\chapter{Conclusion}\label{ch-conc}
This chapter concludes the thesis.
A summary of the major contributions will be given.
Then, an analysis of limitations of this work will help define
a roadmap for follow-on research.
%%%%%%%%%%%%%%%%%%%%%%%%%%%%%%%%%%%%%%%%%%%%%%%%%%%%%%%%%%%%%%%%%%%%%%%%%%%%%%%%%%%%%%%%%%%%%%%%%%%%
\section{Summary of Contributions}
This thesis explored the task of automated white matter hyperintensity segmentation,
which aims to improve the speed and precision over manual analysis.
After reviewing the motivation for this task,
and the previously proposed solutions,
a novel segmentation algorithm -- Voxel-Wise Logistic Regression --
was proposed in Chapter~\ref{ch-intro}.
This algorithm, the necessary regularizations, and pre- and post-processing components
were then developed throughout Chapters~\ref{ch-pre}~and~\ref{ch-vlr}.
Chapter~\ref{ch-exp} then presented extensive validation of
the model and parameter optimization,
including a critical analysis of the popular cross validation frameworks.
In this section, the major contributions and conclusions are summarized.
% ==================================================================================================
\subsection{Algorithm Validation}
One of the major limitations of previous works in this area
is the use of validation conditions which over-estimate
the segmentation performance on images from sources that
were not seen during training.
These conditions include a small number of different image sources,
and the use of training data which comes from the same source as the test data,
an unrealistic condition for many naive algorithm use cases.
In fact, this criticism likely applies to
validation of solutions in many different image analysis tasks, especially MRI.
However, this criticism does assume that the goal of these projects is to develop
algorithms for general-purpose use
-- i.e.\ for segmentation of images from any given source.
It may, in some cases, be possible and preferable to
leverage existing training data for a constant image source
to optimize the algorithm parameters more specifically to these characteristics.
\par
In the current work, the Leave-One-Source-Out Cross Validation (LOSO-CV) framework was presented,
a rediscovery of the ``Multi-Source'' Cross Validation procedure
described by \citeauthor{Geras2013} in~\cite{Geras2013}.
Experimental results showed how other frameworks like
Leave-One-Out (LOO) and K-Fold (KF) Cross Validation
estimate higher segmentation performance than LOSO-CV.
Imprudent use of such frameworks could therefore lead to
premature adoption of particular automated WMH segmentation algorithms,
or overconfidence in their results.
\par
Another fault in previously employed validation frameworks is
the small number of images and different image sources.
Only 5 of the 54 reviewed methods use more than 3 scanners for validation.
Considering the many aspects of image variability, including
MR imaging parameters,
voxel size,
subject anatomy,
scanner field strength, noise and bias field characteristics
(cf.~\S~\ref{ss:autochallenges}),
validation of any brain MRI analysis algorithm
should employ data from more sources which better reflect this variation.
Fortunately, as illustrated here, there are now at least 96 free labelled FLAIR (and T1) image sets
(Table~\ref{tab:database})
which can be used for validation of WMH segmentation algorithms.
Future works in this area are encouraged to follow the lead of the machine learning community
and adopt standardized datasets, facilitating direct comparison of reported performance
across publications.
While the numbers of images from the 7 scanners used in this work are imbalanced
$(20,20,20,5,5,5,21)$,
these data still serve as a good starting point for establishing such a dataset.
% ==================================================================================================
\subsection{Voxel-Wise Logistic Regression}
Many of the previously proposed WMH segmentation algorithms use some sort of thresholding technique,
mapping a single graylevel feature to the class probabilities.
This approach is attractive for FLAIR-only methods,
since it mimics the basic decision making process by human experts.
However, preliminary investigations (Figure~\ref{fig:tpfpfn-thropt}) showed how
such techniques can incur a large number of False Positives and False Negatives,
even if the optimal threshold is used.%
\footnote{The threshold which maximizes Similarity Index with the manual segmentation.}
\par
One circumvention to this problem proposed by \citeauthor{Schmidt2017a}
was to include a ``spatial effect'' parameter $\b^0(x)$,
which expands the feature space from one to four dimensions:
FLAIR graylevel $y$, and spatial coordinates $\xxx$.
However, fitting this model was computationally expensive,
and required various sampling approximations and assumptions about smoothness.
The Voxel-Wise Logistic Regression model overcomes these challenges
by relaxing the assumption that all voxels are equally sensitive to the FLAIR feature
-- i.e.\ that $\b^1$ is constant.
By doing so, all parameters may vary spatially: $\bb \rightarrow \bb(x)$,
yielding independent parameters for all voxels,
and the following predictive model:
\begin{equation}
P\big(c(x)=1\mid\by(x),\bb(x)\big) = \frac{1}{1+e^{-\eta(x)}},\qquad\eta = {\bb(x)}^T\bm{Y}(x).%
\label{eq:eq:vlrmodel-conc}
\end{equation}
In this work, a Newtonian algorithm for estimating the parameters $\bb(x)$ is presented,
maintaining generality for any number of input features.
However, the implementation explored here uses only the FLAIR graylevels,
yielding only two parameter images,
and facilitating the efficient parallel fitting described in \S~\ref{s:parallelfit}.
These images can then be reparametrized via (\ref{eq:reparameterize}) to yield
a threshold image $\mathcal{T}(x)$
and a sensitivity image $\mathcal{S}(x)$.
\par
As explored in \S~\ref{ss:exp-beta}, these parameter images
agree with prior knowledge about the expected class discrimination function.
In particular, the threshold image illustrates a spatially varying graylevel decision boundary
corresponding to $\hat{c}(x) = 0.5$.
As one might expect, this boundary is higher in spatial regions of common false positives
(e.g.\ the cortical gray matter and corpus callosum)
and lower in regions of common false negatives
(e.g.\ the periventricular white matter).
Similarly, the sensitivity image illustrates the separability of the classes,
where lower values arise in voxels with overlapping class distributions.
% --------------------------------------------------------------------------------------------------
\subsubsection{Pre-Processing}
The VLR model assumes that both spatial and graylevel features are comparable across input images,
which is not true for raw MRI.
Standardization of these features, therefore, is the main objective of pre-processing steps.
Graylevel standardization is achieved using a histogram matching operation.
Specifically, it was found that a target histogram with high contrast in the upper intensity range
helped improve segmentation performance.
For standardization of spatial coordinates during model estimation,
images are registered and resampled to the MNI template brainspace
using the \tool{New Segment} feature in SPM.
This confers the additional benefit of estimating and correcting any bias field artifact.
At test time, the inverse transform is applied to map the estimated parameter images
to the native subject space for inference.
% --------------------------------------------------------------------------------------------------
\subsubsection{Regularization}
Estimating independent models for every voxel drastically reduces the number of observations
available to estimate each parameter.
This yields several challenges, namely MLE-fitted parameters which
contradict prior knowledge about the problem, as discussed in cf.~\ref{ss:vlr-chmle}.
To address these challenges, several regularization strategies were explored.
The most successful technique, as in many other problem domains, was data augmentation
(cf.~\S~\ref{ss:exp-reg}).
In this work, this consisted of reasonable image transformations
applied to training image sets, including
reflection about the midline,
and shifting by one voxel (1.5 mm) in MNI space.
In addition, a zero-mean Gaussian prior on the values of $\bb$ was used,
with variance proportional to a tunable parameter $\lambda$.
In order to deal with the unique challenge of an unobserved class (the $c=1$ lesion class)
in many of the peripheral voxels,
pseudo-lesions -- deterministic synthetic data points --
were appended to the training set.
This approach is really just dataset balancing,
and can be employed in other supervised segmentation algorithms.
Finally, parameter images can be smoothed after estimation, using any simple filter;
performance was not particularly sensitive to the selection
(cf.~Figure~\ref{fig:seg-box-beta} in \S~\ref{sss:exp-beta-smooth}).
%%%%%%%%%%%%%%%%%%%%%%%%%%%%%%%%%%%%%%%%%%%%%%%%%%%%%%%%%%%%%%%%%%%%%%%%%%%%%%%%%%%%%%%%%%%%%%%%%%%%
\section{Future Work}
Despite significant segmentation performance gains over its LPA forbearer (cf.~\ref{ss:exp-lpa}),
and performance reaching human level (cf.~\ref{sss:exp-turing}),
there are several limitations to the proposed VLR model.
Chiefly, when training and testing the proposed algorithm on the exact same data,
Similarity Index still only reaches 0.71 (cf. \S~\ref{ss:exp-cv}).
This demonstrates a significant ceiling on performance
which likely cannot be overcome by the regularizations explored in this work:
L2-priors on $\bb$, pseudo-lesions, and parameter image smoothing.
% ==================================================================================================
\subsection{Model Variants}
Potential solutions to this challenge, which might be the subject of future work, include
improved graylevel standardization,
improved image registration, and
inclusion of additional features (graylevel or otherwise).
Improved graylevel standardization is perhaps the most promising,
since there is evidence to suggest that the histogram matching method employed
here was suboptimal, particularly for subjects with large LL.
For example, Recall consistently declined for very high LL
(e.g.\ Figure~\ref{fig:seg-final-scat-re}), as discussed in \S~\ref{ss:exp-finalseg}.
The extent to which registration inaccuracies affected the performance of the VLR model
was not significantly explored.
However, as suggested in \S~\ref{s:pre-reg} and~\cite{Klein2009},
many image registration algorithms have been proposed
which may align anatomical structures better than the SPM \tool{New Segment} method used here.
\par
As outlined in Chapter~\ref{ch-intro}, additional graylevel features were not explored in this work
because the aim was to develop a FLAIR-only WMH segmentation algorithm.
Considering the performance limitations with the current model,
it might be tempting to include graylevel features from other MRI sequences, such as T1 and T2.
However, as noted in \S~\ref{sss:limits-lr}, the WMH class is not monotonic in
T1 or T2 intensities, which makes their inclusion on the logistic regression model improper.
Similarly, texture features, derived from raw MRI sequences could possibly be explored,
but the inclusion of many features significantly erodes the model interpretability,
and there is little evidence to suggest that WMH have unique texture characteristics.
\par
Alternatively, the classification model (currently ``logistic regression'') could be
replaced by a more complex, non-linear model in order to incorporate additional features,
such as a multilayer perceptron~\cite{Hornik1989}.
The contributions of the VLR paradigm to such a model would then derive from 
explicit spatial parametrization of the model parameters,
though this would imply a drastic increase in their number.
This approach is contrasted with methods which employ spatial features (e.g.\ coordinate values)
alongside graylevel features within the segmentation model~\cite{Valverde2017},
and could afford improvements through
differential regularization of conventional and spatial features.
% ==================================================================================================
\subsection{Open Sourcing}
The VLR model may be of interest to other researchers, either
as a clinical research tool for segmenting WMH in large numbers of FLAIR images,
or for future development and improvement, as outlined above.
For this reason, the VLR algorithm will be released in two forms.
First, all associated \matlab\ code will be released through the following public GitHub repository:
\hreftt{github.com/uoguelph-mlrg/vlr}.
This will permit inspection, re-training, and modification of the model
by other medical imaging researchers.
Second, the inference component, including ``pre-trained'' parameter images
from all available data described here,
will be packaged as part of a 3D Slicer\footnote{\hreftt{www.slicer.org/}} module,
providing a graphical user-interface for running the algorithm
without programming experience or access to \matlab.
% ==================================================================================================
\subsection{Deep Learning}
Finally, the poor performance of the proposed model relative to the deep learning approaches
from the 2017 WMH Segmentation Competition (cf.~\S~\ref{ss:exp-wmhseg17}),
especially the U-Net architecture, should not be ignored.
Convolutional neural networks and the VLR model really occupy opposite ends of the
model complexity spectrum.
The VLR model achieves good performance with relatively little fine-tuning,
since prior knowledge of the problem is incorporated in the model itself.
This also provides the main advantage of the VLR model
-- that it is not a ``black box'', that the parameters of the model
illustrate exactly how the classification probability is assigned.
However, as shown here, these ``strongly-biased'' models
often eventually reach a limit in performance due to invalid assumptions.
For example, the VLR model assumes that in a given spatial location,
FLAIR graylevel alone is sufficient to discriminate the WMH class from healthy tissues;
however, in spatial locations where both bright GM and WMH are sometimes observed
due to anatomical variability, this assumption fails.
Instead, more complex contextual information may actually be needed to
make this discrimination.
\par
Deep learning approaches have the capacity to learn such features
much more efficiently than human-guided methods, even at several scales.
Issues associated with learning the complex image-to-image mappings
with such models have recently been overcome with considerable success,
and these methods have clearly emerged
as dominant in other image analysis tasks~\cite{Ronneberger2015}.
The results of the 2017 WMH Segmentation Challenge suggest that this task too
might be added to their palmares.
\par
Therefore, the proposed VLR method may just go the way of the dodo.
The VLR method is, in fact, quite inflexible to application in any other task, due to
constraints about feature monotonicity with the lesion class,
and only two modelled classes.
This is not to say that the work presented here is all for naught.
Many of the regularization strategies explored here,
including pseudo-lesion synthetic data,
parameter constraints (typically already used),
and preprocessing methods for ensuring consistent distributions of input data,
might improve performance of convolutional neural networks.
Indeed, all of the top-performing methods in the 2017 WMH Challenge
already employed standardization of images in space and graylevel dimensions.
Moreover -- perhaps more importantly -- validation of these networks
can and should make use of the LOSO-CV framework presented here.
% --------------------------------------------------------------------------------------------------
% ==================================================================================================
%%%%%%%%%%%%%%%%%%%%%%%%%%%%%%%%%%%%%%%%%%%%%%%%%%%%%%%%%%%%%%%%%%%%%%%%%%%%%%%%%%%%%%%%%%%%%%%%%%%%
