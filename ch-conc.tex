%%%%%%%%%%%%%%%%%%%%%%%%%%%%%%%%%%%%%%%%%%%%%%%%%%%%%%%%%%%%%%%%%%%%%%%%%%%%%%%%%%%%%%%%%%%%%%%%%%%%
% ==================================================================================================
% --------------------------------------------------------------------------------------------------
\chapter{Conclusion}\label{ch-conc}
%%%%%%%%%%%%%%%%%%%%%%%%%%%%%%%%%%%%%%%%%%%%%%%%%%%%%%%%%%%%%%%%%%%%%%%%%%%%%%%%%%%%%%%%%%%%%%%%%%%%
\section{Summary}
%%%%%%%%%%%%%%%%%%%%%%%%%%%%%%%%%%%%%%%%%%%%%%%%%%%%%%%%%%%%%%%%%%%%%%%%%%%%%%%%%%%%%%%%%%%%%%%%%%%%
\section{Future Work}

%The most important data presented in this work are
%the 2017 WMH Segmentation Competition results (cf.~\S~\ref{ss:res-wmhseg17}).
%These results illustrate the performance of the VLR model
%on a database of 110 test subjects from 5 different scanners,
%relative to 19 other state-of-the-art methods.
%In this scenario, the proposed method ranked
%8\ss{th} in SI performance (average 0.70 vs.\ average 0.80 by the first place team)
%and 15\ss{th} in the overall ranking, which considers per-lesion metrics.
%While 
%
%The poor performance of the proposed model relative to
%mainly deep learning approaches (especially the U-Net architecture)
%should not be ignored.
%In all likeliness, the proposed VLR method will go the way of the dodo.
%As the No-CV results indicate (cf.~\S~\ref{ss:res-cv}),
%there are performance limitations associated with this model,
%regardless of regularization techniques or data augmentation.
%For example, in spatial locations where both bright GM and WMH are sometimes observed
%due to anatomical variability,
%the VLR model simply cannot distinguish between these cases.
%Conversely, deep learning approaches could learn local contextual features,
%perhaps at several scales, in order to make this discrimination.
%\par
%Trained deep segmentation models represent a mapping
%from a high dimensional input space
%to a high dimensional output space.
%While end-to-end training schemes have usually yielded performance gains,
%it would certainly be worth investigating potential
%additional gains through preprocessing the data
%to encourage consistency in the input distributions.

%Use SPM segment with the FLAIR + T1 to obtain nice GM/WM segmentation. This allows:
%- mixture-model-based standardization of graylevels, since now more accurate?
%- estimated GM to clean up false positives
%
%deep learning advantages:
%- context-aware
%- minimal graylevel standardization necessary
%The explosion of deep learning approaches for WMH segmentation
%speaks to the advantages of open-source development and publishing
%championed by the deep learning community.
%Graylevel standardization still the key to victory, I think -- need justification of this
%improved registration may also be helpful,
%since~\cite{Klein2009} note that SPM is not the best (time constraints)
% ==================================================================================================
%\subsection{Supervised Graylevel Standardization}
% ==================================================================================================
%\subsection{Integrative Models}
%There are several advantages to the unified models in the SPM and FSL Segment tools.
%In fact, early ambitions for the current work included integrating
%the WMH segmentation model within SPM Segment.
% advantages:
% - better estimation of bias field
% - more robust registration at low voxel resolution.
% - don't include WMH in estimated healthy tissue classes
% why not:
% - time constraints...
% - prelim investigations: using estimated WM / GM distributions for standardization didn't work.
% ==================================================================================================
% --------------------------------------------------------------------------------------------------
% ==================================================================================================
%%%%%%%%%%%%%%%%%%%%%%%%%%%%%%%%%%%%%%%%%%%%%%%%%%%%%%%%%%%%%%%%%%%%%%%%%%%%%%%%%%%%%%%%%%%%%%%%%%%%