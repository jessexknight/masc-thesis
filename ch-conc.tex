%%%%%%%%%%%%%%%%%%%%%%%%%%%%%%%%%%%%%%%%%%%%%%%%%%%%%%%%%%%%%%%%%%%%%%%%%%%%%%%%%%%%%%%%%%%%%%%%%%%%
% ==================================================================================================
% --------------------------------------------------------------------------------------------------
\chapter{Conclusion}\label{ch-conc}
This chapter concludes the thesis.
A summary of the major contributions will be given.
Then, a roadmap for research which builds on this work will be presented,
following an analysis of limitations of this work.
%%%%%%%%%%%%%%%%%%%%%%%%%%%%%%%%%%%%%%%%%%%%%%%%%%%%%%%%%%%%%%%%%%%%%%%%%%%%%%%%%%%%%%%%%%%%%%%%%%%%
\section{Summary}
This thesis explored the task of automated white matter hyperintensity segmentation,
which aims to improve the speed and precision over manual analysis.
% ==================================================================================================
\subsection{Algorithm Validation}
One of the major limitations of previous works in this area
is the use of validation conditions which over-estimate
the segmentation performance on images from sources
were not seen during training.
These conditions include a small number of different image sources,
and the use of training data which comes from the same source as the test data,
an unrealistic condition for most naive algorithm use cases.
In fact, this criticism likely applies to
validation of solutions in many different image analysis tasks.
\par
In the current work, the Leave-One-Source-Out Cross Validation (LOSO-CV) framework is presented,
a rediscovery of the ``Multi-Source'' Cross Validation procedure
described by \citeauthor{Geras2013} in \cite{Geras2013}.
Experimental results show how other frameworks like
Leave-One-Out (LOO) and K-Fold (KF) Cross Validation
estimate higher segmentation performance than LOSO-CV.
Imprudent use of such frameworks could lead to
premature adoption of particular automated WMH segmentation algorithms,
or overconfidence in their results.
%%%%%%%%%%%%%%%%%%%%%%%%%%%%%%%%%%%%%%%%%%%%%%%%%%%%%%%%%%%%%%%%%%%%%%%%%%%%%%%%%%%%%%%%%%%%%%%%%%%%
\section{Future Work}


There are several limitations to the current work.

In \S~\ref{ss:exp-cv} it was shown how the current model
can be trained and tested using the exact same data,
and Similarity Index still only reaches 0.71.
This demonstrates a significant ceiling to performance
which perhaps cannot be overcome through
changes to pre / post-processing or regularization alone.


%The most important data presented in this work are
%the 2017 WMH Segmentation Competition results (cf.~\S~\ref{ss:res-wmhseg17}).
%These results illustrate the performance of the VLR model
%on a database of 110 test subjects from 5 different scanners,
%relative to 19 other state-of-the-art methods.
%In this scenario, the proposed method ranked
%8\ss{th} in SI performance (average 0.70 vs.\ average 0.80 by the first place team)
%and 15\ss{th} in the overall ranking, which considers per-lesion metrics.
%While 
%
%The poor performance of the proposed model relative to
%mainly deep learning approaches (especially the U-Net architecture)
%should not be ignored.
%In all likeliness, the proposed VLR method will go the way of the dodo.
%As the No-CV results indicate (cf.~\S~\ref{ss:res-cv}),
%there are performance limitations associated with this model,
%regardless of regularization techniques or data augmentation.
%For example, in spatial locations where both bright GM and WMH are sometimes observed
%due to anatomical variability,
%the VLR model simply cannot distinguish between these cases.
%Conversely, deep learning approaches could learn local contextual features,
%perhaps at several scales, in order to make this discrimination.
%\par
%Trained deep segmentation models represent a mapping
%from a high dimensional input space
%to a high dimensional output space.
%While end-to-end training schemes have usually yielded performance gains,
%it would certainly be worth investigating potential
%additional gains through preprocessing the data
%to encourage consistency in the input distributions.

%Use SPM segment with the FLAIR + T1 to obtain nice GM/WM segmentation. This allows:
%- mixture-model-based standardization of graylevels, since now more accurate?
%- estimated GM to clean up false positives
%
%deep learning advantages:
%- context-aware
%- minimal graylevel standardization necessary
%The explosion of deep learning approaches for WMH segmentation
%speaks to the advantages of open-source development and publishing
%championed by the deep learning community.
%Graylevel standardization still the key to victory, I think -- need justification of this
%improved registration may also be helpful,
%since~\cite{Klein2009} note that SPM is not the best (time constraints)
% ==================================================================================================
%\subsection{Supervised Graylevel Standardization}
% ==================================================================================================
%\subsection{Integrative Models}
%There are several advantages to the unified models in the SPM and FSL Segment tools.
%In fact, early ambitions for the current work included integrating
%the WMH segmentation model within SPM Segment.
% advantages:
% - better estimation of bias field
% - more robust registration at low voxel resolution.
% - don't include WMH in estimated healthy tissue classes
% why not:
% - time constraints...
% - prelim investigations: using estimated WM / GM distributions for standardization didn't work.
% ==================================================================================================
% --------------------------------------------------------------------------------------------------
% ==================================================================================================
%%%%%%%%%%%%%%%%%%%%%%%%%%%%%%%%%%%%%%%%%%%%%%%%%%%%%%%%%%%%%%%%%%%%%%%%%%%%%%%%%%%%%%%%%%%%%%%%%%%%