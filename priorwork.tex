% --------------------------------------------------------------------------------------------------
% __JK__ need to add Sudre2015 to this and the table!
\priorworksub{Thresholding Techniques}
Since WMH are brighter than healthy brain tissue in FLAIR images,
many unsupervised works have used thresholding of FLAIR intensities
as the initial lesion segmentation.
For example, in the works by%
~\citeauthor{Jack2001}~\cite{Jack2001},%
~\citeauthor{DeBoer2009b}~\cite{DeBoer2009b}, and%
~\citeauthor{Smart2011},~\cite{Smart2011}
optimal FLAIR thresholds are empirically estimated relative to histogram statistics,
though~\citeauthor{DeBoer2009b} use only estimated GM voxels in the histogram.
\citeauthor{Gibson2010} use a conservative FLAIR threshold initially,
but then classify the remaining voxels using Fuzzy C Means clustering~\cite{Gibson2010}.
\citeauthor{Samaille2012} use nonlinear diffusion filtering and watershed segmentation,
before classifying candidate regions based on a FLAIR image threshold.
\citeauthor{Yoo2014} estimate the optimal threshold for FLAIR images using histogram statistics,
derived from a regression model primarily considering the total lesion load~\cite{Yoo2014}.
In works by~\citeauthor{Khademi2014}, a peak in the conditional probability
of edge content on graylevel is used to model partial volume averaging
for unsupervised WML segmentation in FLAIR MRI for subjects with ischemic and MS diseases%
~\cite{Khademi2014,Khademi2015,Knight2016a}.
% --------------------------------------------------------------------------------------------------
\priorworksub{Mixture Models}
Most other unsupervised approaches are probabilistic models, often framed as a mixture model.
The work by~\citeauthor{VanLeemput2001}~\cite{VanLeemput2001} uses
a similar framework as the early work by~\citeauthor{Ashburner1997}~\cite{Ashburner1997},
later incorporated into the SPM ``segment'' tool~\cite{Ashburner2005},
which jointly estimates Gaussian graylevel distributions for each tissue class, and also bias field,
using expectation maximization.
In the model by~\citeauthor{VanLeemput2001},
distribution parameters are estimated using outlier-insensitive estimators,
and WMH are derived from model outliers using heuristic rules.
The predicted classes are also smoothed spatially using a Markov Random Field (MRF).
\par
Similar works by%
~\citeauthor{Bricq2008}~\cite{Bricq2008},%
~\citeauthor{Schmidt2012}~\cite{Schmidt2012},%
~\citeauthor{Jain2015}~\cite{Jain2015}, and%
~\citeauthor{Roura2015}~\cite{Roura2015}
use parametric mixture models to predict WMH as model outliers,
and all but~\cite{Roura2015} embed the model in a MRF.
\citeauthor{Khayati2008}~\cite{Khayati2008} and~\citeauthor{Subbanna2009}~\cite{Subbanna2009}
also use MRF-constrained mixture models, but model WMHs as a Gaussian-distributed tissue class,
rather than as outliers.
In the works by~\citeauthor{Harmouche2006}, parametric distributions are also used to model lesions,
but such distributions are parameterized independently per brain region,
in order to reflect lobe heterogeneity;
a MRF is again used for regularization~\cite{Harmouche2006,Harmouche2015}.
\citeauthor{Schwarz2009} again employ a Bayesian MRF model,
but use lognormal distributions for WM and WMH~\cite{Schwarz2009}.
\citeauthor{Souplet2008} use an augmented mixture model which includes
partial volume averaging classes and an outlier class to perform initial brain tissue segmentation;
WMH are subsequently classified using a FLAIR intensity threshold
after contrast enhancement~\cite{Souplet2008}. 
The work by~\citeauthor{Herskovits2008} is much the same,
but uses statistical information from training data to classify lesions
(i.e.\ it is supervised)~\cite{Herskovits2008}.
More recently, Graph-Cuts have been used in conjunction with mixture models, as in the works by%
~\citeauthor{Garcia-Lorenzo2009}~\cite{Garcia-Lorenzo2009},%
~\citeauthor{Tomas-Fernandez2015}~\cite{Tomas-Fernandez2015}, and%
~\citeauthor{Strumia2016}~\cite{Strumia2016}.
\par
The Lesion-TOADS method by~\citeauthor{Shiee2010}~\cite{Shiee2010},
a lesion-specific adaptation of the TOADS algorithm~\cite{Bazin2008},
presents an entirely new non-Gaussian paradigm for modelling class distributions,
and incorporates topological energies in the objective function.
Other proposed unsupervised methods have used clustering by Fuzzy C-Means, including the works by%
~\citeauthor{Admiraal-Behloul2005}~\cite{Admiraal-Behloul2005},%
~\citeauthor{Gibson2010}~\cite{Gibson2010}, and%
~\citeauthor{Valverde2016}~\cite{Valverde2016}.
% --------------------------------------------------------------------------------------------------
\priorworksub{Classic Supervised Methods}
Many early supervised methods used K-Nearest Neighbours (K-NN) for voxel-wise WMH classification.
\citeauthor{Anbeek2005} used a K-NN model with features derived from
spatial coordinates and voxel intensities from several modalities~\cite{Anbeek2004,Anbeek2005}.
In the works by%
~\citeauthor{Wu2006}~\cite{Wu2006},%
~\citeauthor{Steenwijk2013}~\cite{Steenwijk2013}, and%
~\citeauthor{Fartaria2015}~\cite{Fartaria2015},
spatial coordinates are substituted for tissue priors as K-NN features.
In the recently proposed BIANCA algorithm by~\citeauthor{Griffanti2016}~\cite{Griffanti2016},
spatial coordinates are added back, along with some patch-based features.
\par
Other works have also explored Support Vector Machines (SVM) for classification.
The works by%
~\citeauthor{Lao2006}~\cite{Lao2006},%
~\citeauthor{Abdullah2012}~\cite{Abdullah2012}, and%
~\citeauthor{Scully2010}~\cite{Scully2010}
each use a selection of
intensity features, neighbouring intensities, tissue priors, morphological, and texture features
with an SVM classifier.
Several more recent works have used decision tree-based classifiers,
including Random Forest (RF) and AdaBoost.
\citeauthor{Akselrod-Ballin2009}~\cite{Akselrod-Ballin2009} employ over 30 features
for multi-scale image representation and classify voxels using RF.
Both~\citeauthor{Geremia2011}~\cite{Geremia2011} and~\citeauthor{Roy2015}~\cite{Roy2015}
use a combination of intensity and tissue prior features to train a RF classifier,
whereas~\citeauthor{Wels2008}~\cite{Wels2008} use a large number of Haar-like features
to train an AdaBoost model.
\citeauthor{Ithapu2014}~\cite{Ithapu2014} explore the use of
texton features in both SVM and RF models.
\par
Logistic regression models have also gained popularity recently.
In the OASIS model by~\citeauthor{Sweeney2013}~\cite{Sweeney2013},
image intensities from T1, T2, PD, and FLAIR sequences are used
individually, in multiplicative combination, and with Gaussian blurring
as predictors for a global set of logistic regression parameters.
In the work by~\citeauthor{Zhan2017}~\cite{Zhan2017}, a similar logistic model is fitted using
only the raw T1, T2, and FLAIR intensities, while
bias correction is performed as preprocessing and spatial smoothness using MRF post processing.
In the work by~\citeauthor{Dadar2017}~\cite{Dadar2017},
spatial and intensity features from a flexible selection of MR sequences are used to
train a linear regression model, the results of which are thresholded to give the lesion prediction.
Still more works have proposed other supervised models,
including nonparametric Parzen classifiers~\cite{Sajja2006}.
% --------------------------------------------------------------------------------------------------
\priorworksub{Deep Learning}
A number of deep learning approaches have also been proposed,
though their permeation in this problem space has been surprisingly limited until recently%
\footnote{The 2017 WMH Segmentation Competition, saw a massive increase, however, with
  15/20 submitted methods using deep learning;
  cf.~\S~\ref{ss:exp-wmhseg17} for more information.}.
Both~\citeauthor{Zijdenbos2002}~\cite{Zijdenbos2002} and~\citeauthor{Dyrby2008}~\cite{Dyrby2008}
train fully-connected voxel-wise Neural Networks with a selection of
intensity, spatial, and tissue prior features to predict the lesion class.
In contrast,~\citeauthor{Brosch2015}~\cite{Brosch2015,Brosch2016} construct a more modern
deep convolutional model, which is capable of capturing both local and global dependencies.
% --------------------------------------------------------------------------------------------------
\priorworksub{External Toolboxes}
Many of the proposed methods use registration, brain extraction, bias field correction, and segmentation tools available in freely available toolkits; these include the SPM%
\footnote{\hreftt{http://www.fil.ion.ucl.ac.uk/spm/}} toolkit
~\cite{Sajja2006,Dyrby2008,Akselrod-Ballin2009,Smart2011,Schmidt2012,Yoo2014,Ithapu2014,Roy2015,Valverde2016} and the
FSL%
\footnote{\hreftt{https://fsl.fmrib.ox.ac.uk/fsl/}} toolkit
~\cite{Herskovits2008,Gibson2010,Datta2013,Steenwijk2013,Sweeney2013,Roy2015,Wang2015,Griffanti2016,Zhan2017},
as well as bias correction by the
N3/4%
\footnote{\hreftt{https://www.slicer.org/wiki/Documentation/4.6/Modules/N4ITKBiasFieldCorrection}} 
~\cite{Tustison2010}
algorithm
~\cite{Zijdenbos2002,Harmouche2006,Fartaria2015,Guizard2015,Harmouche2015,Mechrez2016,Valverde2016,Dadar2017,Zhan2017}.
% --------------------------------------------------------------------------------------------------